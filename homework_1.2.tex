\errorcontextlines=8

\documentclass{article}

\widowpenalty=10000
\clubpenalty=10000

\usepackage{fullpage}
\usepackage{fancyhdr}
\pagestyle{fancy}

\headheight=14pt
\headsep=16pt
\makeatletter
\lhead{\@date}
\chead{\@title}
\rhead{\@author}
\makeatother

\parindent=0pt
\parskip=6pt

\usepackage{alltt}
\newenvironment{pseudocode}
  {\begin{quote}\begin{alltt}\normalfont}
  {\end{alltt}\end{quote}}

\usepackage{enumerate}
\usepackage{color}
\usepackage{rotating}
\usepackage{epsfig}  %EPS graphic package
\usepackage{amsmath} %AMSMath font package
\usepackage{amssymb} %AMSMath symbol package
\usepackage{amsthm}  %Mathematical environment package
\usepackage{booktabs}   %Tables package
\usepackage[usenames,dvipsnames,svgnames,table]{xcolor} %Colors package

%Define commands
\newcommand{\problem}{\vspace{.5cm}\textbf{Problem:~~}}
\newcommand{\answer}{\vspace{.25cm}\emph{Answer:~~}}

\begin{document}

\title{Homework~1.2}
\author{Brooke Lampe}
\date{August~24,~2017}

%problems
\textbf{Problem 1.}

	\begin{enumerate}[a.]
		\item $\sum_{i=0}^{n - 1} 12i = $
		
			$ = 12 \sum_{i=0}^{n - 1} i$
		
			$ = 12 \times (i(i - 1))/2 \rvert_{0}^{n}$
		
			$ = (12n(n - 1)) / 2 - (12(0)(0 - 1)) / 2$
		
			$ = 6(n^2 - n)$
		
		\item $\sum_{i=0}^{n - 1} 2^{i + 2} = $
		
			$ = \sum_{i=0}^{n - 1} 2^i \times 2^2$
		
			$ = 4 \sum_{i=0}^{n - 1} 2^i$
			
			$ = 4 (2^i / (2 - 1)) \rvert_{0}^{n}$
			
			$ = 4 (2^i) \rvert_{0}^{n}$
			
			$ = 4 (2^n) - 4 (2^0)$
			
			$ = 4 (2^n - 1)$
		
		\item $\sum_{i=0}^{n - 1} \sum_{j = i}^{n - 1} 1 = $
		
			$ = \sum_{i=0}^{n - 1} j \rvert_{i}^{n}$
		
			$ = \sum_{i=0}^{n - 1} (n - i)$
		
			$ = n \sum_{i=0}^{n - 1} 1 - \sum_{i=0}^{n - 1} i$
		
			$ = n(i) \rvert_{0}^{n} - (i(i - 1))/2 \rvert_{0}^{n}$
			
			$ = n(n) - (n(n - 1))/2$
			
			$ = (1/2)(n^2 + n)$
	\end{enumerate}

\newpage

\textbf{Problem 2.}

	Let $f(n)$ be $10n^2 + 100n + 1000$.  Let $g(n)$ be $n^2$.
	
	$f(n) \in \Theta(g(n))$ if and only if $f(n) \in \Omega(g(n))$ and $f(n) \in O(g(n))$.
	
	Show that $f(n) \in \Omega(g(n))$
	
	\hspace{4ex} $f(n) \geq c \times g(n)$ for all $n > N$
	
	\hspace{4ex} Let $c = 1$ and $N = 0$
	
	\hspace{4ex} $f(n) \geq 1 \times g(n)$  for all $n > 0$
		
	\hspace{4ex} $10n^2 + 100n + 1000 \geq 1 \times n^2$ for all $n > 0$
	
	\hspace{4ex} $10(0)^2 + 100(0) + 1000 \geq 1 \times (0)^2$
	
	\hspace{4ex} $1000 \geq 0$
	
	Show that $f(n) \in \Theta(g(n))$
	
	\hspace{4ex} $f(n) \leq c \times g(n)$ for all $n > N$
	
	\hspace{4ex} Let $c = 10000$ and $N = 1$
	
	\hspace{4ex} $f(n) \leq 10000 \times g(n)$  for all $n > 1$
	
	\hspace{4ex} $10n^2 + 100n + 1000 \leq 10000 \times n^2$ for all $n > 1$
	
	\hspace{4ex} $10(1)^2 + 100(1) + 1000 \leq 10000 \times (1)^2$
	
	\hspace{4ex} $1110 \leq 10000$
	
	$f(n) \in \Omega(g(n))$ and $f(n) \in O(g(n))$; therefore, $f(n) \in \Theta(g(n))$.
	
\newpage
	
\textbf{Problem 3.}

		\begin{enumerate}[a.]
			\item $\sum_{i=0}^{n - 1} 2i^3 - 3i^2 + i = $
			
				$ = \sum_{i=0}^{n - 1} 2i_{(3)} + (2i^3 - 2i_{(3)}) - 3i^2 + i$
				
				$ = \sum_{i=0}^{n - 1} 2i_{(3)} - 9i^2 + 5i$
				
				$ = \sum_{i=0}^{n - 1} 2i_{(3)} - 9i_{(2)} + (-9i^2 + 9i_{(2)}) + 5i$
				
				$ = \sum_{i=0}^{n - 1} 2i_{(3)} - 9i_{(2)} - 4i$
				
				$ = \sum_{i=0}^{n - 1} 2i_{(3)} - 9i_{(2)} - 4i_{(1)}$
				
				$ = 2i_{(4)}/4 - 9i_{(3)}/3 - 4i_{(2)}/2 \rvert_{0}^{n}$
				
				$ = 0.5i^4 - 6i^3 + 12.5i^2 - 7i \rvert_{0}^{n}$
				
				$ = 0.5n^4 - 6n^3 + 12.5n^2 - 7n$
				
				Order of growth is $\Theta(n^4)$
				
			\item $\sum_{i=1}^{n} (4 + log_{2} (i)) = $
			
				$ = \sum_{i=1}^{(n + 1) - 1} 4 + \sum_{i=1}^{n} log_{2} (i))$
				
				$ = 4i \rvert_{1}^{n + 1} + log_{2} (n!)$	\hfill Since $log_{2}(a) + log_{2}(b) = log_{2}(a \times b)$, we have $log_{2} (n!)$
				
				$ = 4n + log_{2} (n!)$
				
				Order of growth is $\Theta(log_{2} (n!))$
		\end{enumerate}
	
\textbf{Problem 4.}

	To prove that, for all distinct $a$ and $b$ greater than one, $a^n \neq \Theta(b^n)$, we use the limit method.
	
	$\lim_{n\to\infty} a^n / b^n = \lim_{n\to\infty} (a / b)^n$
	
	Looking at the simplified expression, it is clear that if $a$ is greater than $b$ and both $a$ and $b$ are greater than one, then the limit will diverge to infinity.  If $b$ is greater than $a$ and both $a$ and $b$ are greater than one, then the limit will converge to zero.  For two functions to be $\Theta$ of each other, the limit must approach a non-zero constant.  Neither diverging to infinity nor converging to zero approaches a non-zero constant.  Furthermore, if we look at $\lim_{n\to\infty} (a / b)^n$, we can see that the limit will only approach a constant if $a$ and $b$ are equal, allowing the limit to approach $1$.  But in this proof, we are only concerned with distinct $a$ and $b$, so it is not possible to obtain a constant factor through the limit method; thus, it is not possible to have two distinct $a$ and $b$ be $\Theta$ of each other.
	
\textbf{Problem 5.}

	To prove that, for all distinct $a$ and $b$ greater than one, $log_{a} (n) = \Theta(log_{b} (n))$, we use the change of base formula.
	
	$log_{a} (n) = log_{b} (n) / log_{b} (a)$
	
	$log_{a} (n) = log_{b} (n) \times (1 / log_{b} (a))$
	
	The difference between $log_{a} (n)$ and $log_{b} (n)$ is a constant factor $(1 / log_{b} (a))$.  Because $(1 / log_{b} (a))$ is a constant factor, it will not impact the rate of growth of the function; functions that are within a constant factor of each other are $\Theta$ of each other.  Thus, we have $log_{a} (n) = \Theta(log_{b} (n))$.
	
\newpage

\textbf{Problem 6.}

	\begin{enumerate}[a.]
		\item $n$ is the number of elements in the list.
		
		\item The incrementation of $s$ ($s \leftarrow s + 1$) in lines 5, 9, and 14 is the basic operation.
		
		\item In the loop at line 4, the elementary operation is executed $\lfloor log_{2}(n) \rfloor$ times.
		
			In the loop at line 8, the elementary operation is executed $n$ times.
			
			Combined, the line 4 and line 8 loops execute the elementary operation $\lfloor log_{2}(n) \rfloor + n$ times.
			
			The line 4 and line 8 loops are within the loop at line 2, which executes $n$ times.
			
			So, the line 4 and line 8 loops are executed $n$ times, which executes the elementary operation $n \times (\lfloor log_{2}(n) \rfloor + n)$ times.
			
			Asymptotically, the floor function does not have a significant impact, so we can omit it and simplify.
			
			From line 1 through line 11, the elementary operation is executed $n^2 + n$ $log_{2}(n)$ times.
			
			Regardless of the value of $j$, the loop in line 13 executes 3 times $(j, j + 1, j + 2)$, and the elementary operation is executed once each time.
			
			The loop in line 13 is executed by the loop in line 12, which executes $n$ times.
			
			Thus, the elementary operation in line 14 is executed $3n$ times by the loops in line 12 and line 13.
			
			In total, the elementary operation is executed $n^2 + n$ $log_{2}(n) + 3n$ times.
			
			Asymptotically, we consider only the fastest growing term, which is $n^2$.
			
			Thus, the asymptotic time complexity is $\Theta(n^2)$.
	
		\item The loops in lines 2, 8, and 12 all execute $n$ times.  However, the loop in line 2 contains two additional loops that must be completely executed every time the loop in line 2 executes.  The loop in line 2 causes the elementary operation to be executed $\Theta(n^2)$ times.  Thus, the loop in line 2 can be considered the bottleneck loop, which will dominate the runtime for large $n$.
	\end{enumerate}

\newpage

\textbf{Problem 7.}

	Basic operations:  Comparison of elements, comparison of element products
	
	Time complexity of first implementation:
	
	\hspace{4ex} $2n^2 - n$
	
	\hspace{4ex} element comparison and element product comparison in line 4, which will occur $2n$ times in the inner loop, which will occur $n$ times in the outer loop ($n^2$).  However, the second comparison does not execute when the indexes are the same ($-n$).
	
	\hspace{4ex} $\Theta(n^2)$
	
	Time complexity of second implementation:
	
	\hspace{4ex} $\sum_{i=0}^{n - 1} \sum_{j = i}^{n - 1} 1 = (1/2)(n^2 + n)$
	
	\hspace{4ex} element comparison in line 4
	
	\hspace{4ex} $\Theta(n^2)$
	
	Time complexity of third implementation:
	
	\hspace{4ex} $n$ $log(n) + 1$
	
	\hspace{4ex} sorting and element product comparison in line 5
	
	\hspace{4ex} $\Theta(n$ $log(n))$
	
	\begin{enumerate}[a.]
		\item Both of the suggestions are faster than the original for non-trivial $n$.  The original will run approximately $2n^2$ comparisons while the first suggestion will run $(1/2)n^2$ comparisons.  The third comparison will run even faster than either the original or the first suggestion, at $n$ $log(n)$ time instead of $n^2$ time.
		
		\item The original and the first suggestion are asymptotically the same ($n^2$ time).  The second suggestion is faster than the original and the first suggestion because it runs in quasilinear time ($n$ $log(n)$ time), rather than quadratic time ($n^2$ time).
		
		\item The second suggestion requires the list to be sorted, but is does not require an ascending or descending sort.  An actual implementation could sort either direction, so the pseudocode allows for both sorting implementations by assuming that the product of the final two elements is greatest (ascending order), and then using a conditional to check if the product of the first two elements is greatest (descending order), and respond appropriately.
	\end{enumerate}

\newpage

\textbf{Problem 8.}

\textbf{Input: A,} a list

\hspace{4ex}	\textbf{if} $|A| > 1$

\hspace{8ex}		$a \leftarrow 0$

\hspace{8ex}		$b \leftarrow 0$

\hspace{8ex}		\textbf{for} $i \in 0 ... |A| - 1$ \textbf{do}

\hspace{12ex}	 		\textbf{if} $A[a] < A[i]$ \textbf{then}

\hspace{16ex}				$a \leftarrow i$

\hspace{8ex}		\textbf{if} $a = 0$ \textbf{then}

\hspace{12ex}			$b \leftarrow 1$

\hspace{8ex}		\textbf{for} $i \in 0 ... |A| - 1$ \textbf{do}

\hspace{12ex}	 		\textbf{if} $A[b] < A[i] \wedge i \neq a$ \textbf{then}

\hspace{16ex}				$b \leftarrow i$

\hspace{8ex}		\textbf{return} $A[a] \times A[b]$

\textbf{Problem 9.}

	Input: $n$ = $|A|$
	Basic operation:  The comparison in line 3.

	\begin{enumerate}[a.]
		\item When $n$ is zero or one, the while loop in line 2 is never entered, so the basic operation is never executed.
		
			Best case: $\Theta(0)$
			
		\item When $n$ is greater than one and no element in $A$ has a value of zero, the basic operation will be executed $n$ times.
		
			Worst case: $\Theta(n)$
	\end{enumerate}

\textbf{Problem 10.}

	Basic operation: The assignment operation of $i$ (lines 1 and 3 in both algorithms)
	
	In the first algorithm, in the worst case, the \textbf{random()} function happens to draw a random number that is equal to $a$, which requires that the loop in line 2 be repeated.  Though very improbable, in the worst case, this could continue indefinitely, with the random number chosen happening to match $a$ each time.  Worst case of the first algorithm is that the basic operation is executed an infinite number of times: $\Theta(\infty)$
	
	In the second algorithm, in the worst case, the random number selected is greater than or equal to $a$, which causes the basic operation inside the conditional to be executed.  Worst case of the second algorithm is 2 executions of the basic operation (once in line 1, once in line 3): $\Theta(2)$.
	
	The second algorithm has much better worst case performance than the first algorithm.

\end{document}