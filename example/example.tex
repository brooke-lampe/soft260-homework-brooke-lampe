% LaTeX variables and commands usually start with a backslash (\):
%     \example
% Characters like a tilde (~), underscore (_), or caret (^) are also commands.
% Arguments longer than one character are written in curlies ({ and }):
%     \example{first argument}{second argument}

% A LaTeX file begins with a prelude containing imports and settings.
% The actual content starts after the \begin{document} (on line 52).

% Show up to 8 layers of macro expansion (backtrace) when an error is hit.
% Atom will hide these details, but they'll still appear in the log file.
\errorcontextlines=8

% Choose a document class.
\documentclass{article}

% Don't allow widows or club lines (a.k.a. orphans).
\widowpenalty=10000
\clubpenalty=10000

% Use full pages with headers and footers.
\usepackage{fullpage}
\usepackage{fancyhdr}
\pagestyle{fancy}

% Base the header on the document's date, title, and author.
\headheight=14pt
\headsep=16pt
\makeatletter
\lhead{\@date}
\chead{\@title}
\rhead{\@author}
\makeatother

% Set up paragraph formatting.
\parindent=0pt
\parskip=6pt

% Provide a lightweight pseudocode environment.
\usepackage{alltt}
\newenvironment{pseudocode}
  {\begin{quote}\begin{alltt}\normalfont}
  {\end{alltt}\end{quote}}

% Inport common math commands.
\usepackage{amsmath}
\usepackage{amssymb}


% The part between \begin{document} and \end{document} is the actual content.
% Make your changes here.
\begin{document}

% Set the title, author, and date.
% If you leave out the date command, LaTeX will use the current date.
\title{Homework~X.Y}
\author{Ada~Lovelace}
\date{August~XX,~20YY}

% \paragraph creates a paragraph caption; you can use it for problem numbers.
% Blank lines separate paragraphs.
\paragraph{1.} This is a paragraph.

This is another paragraph.
This is another sentence in the second paragraph.

% Content between \( and \) is typeset as inline math.
% Content between \[ and \] is typeset as display math.
% \le is ≤, \ge is ≥, and \ne is ≠.
% \cdot is · and \frac creates a fraction.
% We use extra spaces so LaTeX doesn't think j is part of a command name.
\paragraph{2.} Relational operators: \(i<j\), \(i\le j\), \(i=j\), \(i\ge j\),
\(i>j\), and \(i\ne j\).  Nonrelational binary operators:

\[a+b-c\cdot\frac{d}{e}\]

% \log creates a logarithm.
% \sum creates a summation.
% \int creates an integral, and \, creates the space before its differential.
% \left. and \right\rvert surround an evaluated-at expression.
% \lim creates a limit, and \to is "goes-to", normally \rightarrow (→).
% _ creates a subscript and ^ creates a superscript.
\paragraph{3.} Some expressions: \(\log_2 n\), \(\sum_{i=0}^{n-1}i^2\),
\(\int_{-10}^{10}x\,dx\), \(\left.\frac{x^2}{2}\right\rvert_{-10}^{10}\), and
\(\lim_{x\to\infty}x\).

% \Theta and \Omega are capital Greek letters.
% \left( and \right) are parentheses that grow to match their contents.
\paragraph{4.} Asymptotics: \(O\left(\frac{n}{\log n}\right)\),
\(\Theta\left(\frac{n}{\log n}\right)\), and
\(\Omega\left(\frac{n}{\log n}\right)\).

% \text places text inside of math; just \(foo\) would be f times o times o.
% \varnothing is Ø.
% \{ and \} are curly braces.
% \ldots is ….
% \left\lvert and \right\rvert are vertical bars for cardinality.
\paragraph{5.} \(\text{foo}(0)=\varnothing\) and
\(\left\lvert\{0\ldots n-1\}\right\rvert=n\).

\paragraph{6.} This algorithm prints the numbers from one to 21, inclusive:

% Using this template, \begin{pseudocode} and \end{pseudocode} will format
%   simple pseudocode.  More advanced pseudocode packages like algorithm2e are
%   also available.
% \in is ∈.
% \gets is an assignment, normally \leftarrow (←).
\begin{pseudocode}
  for \(i\in[0\ldots 20]\):
    \(j\gets i+1\)
    \(\text{print}(j)\)
\end{pseudocode}

\end{document}
